\documentclass[11pt,a4paper]{moderncv}

\usepackage{PierreKancir}
\moderncvtheme[blue]{classic}
\usepackage[utf8]{inputenc}       % encodage à privilégier pour la portabilité et +
\usepackage[T1]{fontenc}      % encodage à privilégier pour la portabilité et +
\usepackage{eurosym}
\usepackage[english]{babel}
\usepackage{lmodern} 
\usepackage[margin=2.5cm]{geometry}
%\frenchbsetup{StandardLists=true} % regle les problemes de lists
% \linespread{0.9}
\AfterPreamble{\hypersetup{
  pdfauthor={Pierre Kancir},
  pdftitle={Pierre Kancir CV},
  pdfsubject={Pierre Kancir CV},
  colorlinks=true,
  urlcolor=blue,
  linkcolor=blue,
}}

\usepackage[firstyear=2007,lastyear=2020]{moderntimeline}
%\tlmaxdates{1999}{2012}
% \tlwidth{0.8ex}

\title{Robotics engineer}
\extrainfo{%
\linkedinsocialsymbol~\httplink{www.linkedin.com/in/pierre-kancir/en}\\%
\githubsocialsymbol~\httplink{www.github.com/khancyr}\\%
%Permis B%
}

% personal data
\firstname{Pierre}
\familyname{Kancir}
\address{5 rue Robelin}{35000 Rennes}
\mobile{+33 (0) 6 67 64 64 84}
%\phone{}
%\fax{fax (optional)}                          % optional, remove the line if not wanted
\email{pierre.kancir@gmail.com}
%\homepage{}
%\photo[92pt]{identity}


%\nopagenumbers{}                             % uncomment to suppress automatic page numbering for CVs longer than one page
%----------------------------------------------------------------------------------
%            content
%----------------------------------------------------------------------------------
\begin{document}

%\setmainfont{Minion Pro}
%\setsansfont{Myriad Pro}

\hyphenpenalty=10000

\maketitle

\renewcommand{\labelitemi}{-- }

\section{Professional experiences}

\cventry{2020 -- Today}{Freelance in Robotics}{\href{https://www.hivebotics.fr}{Hivebotics}}{Rennes}{France}{%
Mobile robotics engineering office.
\begin{itemize}
    \item Software development.
    \item Consulting in robotics : testing, code and product reviews, technological watch.
\end{itemize}}

\cventry{2019 -- 2019}{R\&D Engineer}{\href{https://www.esoftthings.com/en/}{esoftthings}}{Cesson-Sévigné}{France}{%
SME specialised in the design of Advanced Driver-Assistance Systems (ADAS) based on vision and embedded artificial intelligence.
\begin{itemize}
    \item Creation and evaluation of an algorithm for estimating the distance of detections made on an electronic mirror.
    \item Writing of a user manual for the Ambarella CV22 card.
    \item Implementation of a Driver Monitoring System algorithm on the CV22 card for a demonstration.
\end{itemize}}

\cventry{2017 -- 2019}{R\&D Engineer}{\href{http:///www.azurdrones.com/}{AzurDrones / Skeyetech}}{Mérignac}{France}{%
SME specialising in data capture and processing services, consulting, training and surveillance by UAVs.
\begin{itemize}
 \item Creation of the first version of \href{https://www.youtube.com/watch?v=TsEw0Ob9nrs}{the autonomous UAV solution with docking station} and production start-up ; 
 \item Bringing up of the embedded system department: project management, installation of versioning, continuous integration and testing tools. Supervision of the transition from 4 to 10 people;
 \item \href{https://www.youtube.com/watch?v=PKJwuSgr23o&list=PL6sCNLbHuYxaLtItzADD5zHxixUtQPSvk&index=7&t=0s}{Strengthening through the use of simulations} of the management and fault detection and navigation methods of the flying UAV;
 \item Definition, implementation and testing of system redundancies;
 \item Management of the technological choices of the autonomous UAV project to obtain the first authorisation for autonomous unmanned flight in France, start of research projects for the second version.
\end{itemize}}

\cventry{2013 -- 2017}{PhD Student/ R\&D Engineer}{\href{http://www.retdtechfrance.fr/}{Retdtechfrance}}{Rennes}{France}{%
Startup specialised in research and development in mechatronics and manufacturer of drones and robotic systems.
\begin{itemize}
    \item Creation of robotic systems, multi-robots platform systems (air, land, sea);
    \item Creation of HMIs with web technologies for robot control and real-time video display;
    \item Multi partners research projects: UAV swarm {\href{https://recherche.telecom-bretagne.eu/daisie}{DAISIE}}, Inflatable robot arm {\href{http://www.warein-sas.com/c/innovation}{Bulle}}.
\end{itemize}}

\pagebreak

\section{Competencies}

%\subsection{Robotique}
\cvcomputer{Embedded systems}{Linux SBC type : Rasberry Pi, Odroid; Microcontrollers : Esp8266, Arduino, AVR, STM32; Android}
\cvcomputer{Systems}{Control theory, communication, localisation, navigation, on-board intelligence, Human-Machine Interface (HMI)}
\cvcomputer{Platforms}{UGV (Unmanned Ground Vehicle), UAV (Unmanned Aerial Vehicle), IOT}
\cvcomputer{Simulators}{\href{http://ardupilot.org/dev/docs/sitl-simulator-software-in-the-loop.html}{ArduPilot SITL}\developer, \href{http://gazebosim.org/}{Gazebo}\contributor}

%\subsection{Développement}
\cvcomputer{Languages}{C, C++, Python, Shell/Bash/Zsh, GNU Make, JavaScript}
\cvcomputer{Frameworks}{\href{https://www.ros.org/}{ROS}\contributor, \href{http://ardupilot.org}{ArduPilot}\developer, OPENCV, HTML, Bootstrap, Websocket, Matlab}
\cvcomputer{Tools}{Git, Mercurial, Cmake, Waf, GitHub, Gitlab, Clion, Docker, Travis}
\cvcomputer{Methods}{TDD, Kanban, CI(Continuous Integration)}
\cvcomputer{Operating Systems}{GNU/Linux (Debian, Ubuntu), Chibios, Windows, Android}

\devnotes{Developer}{Contributor}


\section{Diplomas}

\tlcventry{2013}{2018}{PhD CIFRE-Défense : Multi-robot system design methodology: from simulation to demonstration}{\href{http://www.lab-sticc.fr/en/index/}{Labsticc/Retdtech France}}{Rennes}{\href{https://drive.google.com/open?id=1kunx6TPXdK-vvV7pQeexgF_9M8Roi05N}{Manuscrit}}{}

\tllabelcventry{2010}{2013}{2010 -- 2013}{Engineering Degree}{\href{https://www.imt-atlantique.fr/}{école des Mines de Nantes/ IMT Atlantique}}{Nantes}{Option: Control theory and Industrial software}{}

%\tllabelcventry{2007}{2010}{2007 -- 2010}{Classe préparatoire scientifique aux grandes écoles en filière Physique et Technologie}{\href{http://www.ldmraspail.fr/cpge-ptsi-pt.php}{Lycée Raspail}}{Paris}{Classe Etoile}{}

%\tldatecventry{2007}{Baccalauréat S}{\href{http://www.lm-st-cyr.fr/}{Lycée Militaire Saint Cyr l'école}}{Saint Cyr l'école}{
%Mention Assez Bien
%}{}


\subsection{Other Experiences}
\cventry{2018 - 2020}{Google Summer Of Code}{\href{https://summerofcode.withgoogle.com/archive/2018/projects/5702907169079296/}{Google}}{Mentor}{}{%
%Encadrement d'un étudiant au GSOC 2018 avec ArduPilot sur un projet de suivi de chemin par apprentissage avec un réseau de neurone pour un drone volant.
}

\cventry{2017}{Google Summer Of Code}{\href{https://summerofcode.withgoogle.com/projects/\#6723821496172544}{Google}}{Student}{}{%
Participation in the GSOC 2017, open source software development program set up by Google with competitive selection, with ArduPilot.
%\begin{itemize}
%    \item Soumission d'un projet et acceptation d'un projet d'amélioration du support des drones terrestre (UGV).
%	\item Réécriture du code du Rover pour accepter différentes configurations et types de moteurs ;
%	\item Implémentation de l'évitement d'obstacles ;
%	\item Réécriture de l'intégration continue.
%\end{itemize}
}

\cventry{2015 -- Today}{Developer ArduPilot}{\href{http://ardupilot.org}{ArduPilot}}{}{}{%
ArduPilot is an open-source project of autopilot for multi-platform micro vehicles.
\begin{itemize}
	\item Code improvement for UGV and UAV;
 	%\item Réécriture d'une partie du code en C++11 et Python3 et corrections de bugs ;
 	\item Improved the SITL (Software In The Loop) simulator and automated tests for continuous integration;
 	\item Improved compatibility with ROS and Gazebo;
 	\item Community support and participation in project governance.
\end{itemize}}

\section{Languages}
\cvlanguage{French}{Mother tongue}{}
\cvlanguage{English}{Fluent, IELTS: 7.0/9.0}{}
\cvlanguage{Portuguese}{Beginner}{}
%\cvlanguage{Russe}{Débutant}{étudié à l'école et nombreux voyages dans les pays russophones}

\section{Areas of interest}
\cvhobby{Cooking}{Lovers of good food both at home and outdoors}
\cvhobby{Do-it-yourself}{Home staging, homelab and miscellaneous home improvements}

\end{document}

