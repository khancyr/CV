\documentclass[11pt,a4paper]{moderncv}

\usepackage{PierreKancir}
\usepackage[french]{babel}
%\frenchbsetup{StandardLists=true} % regle les problemes de lists
\linespread{0.9}

\title{Ing\'enieur Robotique}
\extrainfo{%
\linkedinsocialsymbol~\httplink{www.linkedin.com/in/pierre-kancir/fr}\\%
%\githubsocialsymbol~\httplink{www.github.com/khancyr}\\%
%Permis B%
}


%\nopagenumbers{}                             % uncomment to suppress automatic page numbering for CVs longer than one page
%----------------------------------------------------------------------------------
%            content
%----------------------------------------------------------------------------------
\begin{document}

%\setmainfont{Minion Pro}
%\setsansfont{Myriad Pro}

\hyphenpenalty=10000

\maketitle

\renewcommand{\labelitemi}{-- }

\section{Exp\'erience professionnelle}

\cventry{2013 -- Aujourd'hui}{Doctorant/ Ing\'enieur d'\'etude}{\href{http://www.retdtechfrance.fr/}{Retdtechfrance}}{Rennes}{}{%
TPE sp\'ecialis\'ee en recherche et d\'eveloppement en m\'ecatronique et constructeur de drones et de syst\`emes robotiques.
\begin{itemize}
 \item Cr\'eation de syst\`emes robotiques et robotis\'es, syt\`emes multi robots multi-plateformes (air, terre, mer) ;
 \item Cr\'eation d'IHM avec des technologie Web pour la commande des robots et l'affichage de vid\'eo en temps r\'eel ;
 \item Gestion de projet, r\'eponses \`a consultations, \`a appels d'offres publics, r\'edaction de documentation technique ;
 \item Participation \`a des projets de recherches: Essaim de drones {\href{https://recherche.telecom-bretagne.eu/daisie}{DAISIE}}, Bras robot gonflable {\href{http://www.warein-sas.com/c/innovation}{Bulle}}
\end{itemize}}

\subsection{Autres Exp\'eriences}
\cventry{2017 -- Aujourd'hui}{Google Summer Of Code}{\href{https://summerofcode.withgoogle.com/projects/\#6723821496172544}{Google}}{}{}{%
Participation au GSOC 2017 avec Ardupilot.
\begin{itemize}
	\item R\'eecriture du code du Rover pour accepter diff\'erentes configurations et types de moteurs ;
	\item Impl\'ementation de l'\'evitement d'obstacles ;
	\item Am\'elioration de la compatibilit\'e avec ROS et Gazebo et cr\'eation de tutoriels ;
	\item R\'eecriture de l'int\'egration continue.
\end{itemize}}
\cventry{2015 -- Aujourd'hui}{Developpeur ArduPilot}{\href{ardupilot.org}{Ardupilot}}{}{}{%
Ardupilot est un projet open-source d'autopilote pour micro v\'ehicules multi-plateformes.
\begin{itemize}
	\item Am\'elioration du code pour UGV ;
 	\item R\'eecriture d'une partie du code en C++11 et Python3 et bugs fixes ;
 	\item Am\'elioration du simulateur SITL (Software In The Loop) ;
 	\item Am\'elioration de la compatibilit\'e avec ROS et Gazebo ;
 	\item Aide \`a la communaut\'e et participation aux directions prises par le projet.
\end{itemize}}



\section{Comp\'etences}

\subsection{Robotique}
\cvcomputer{Syst\`emes embarqu\'es}{Type Rasberry Pi : Rasberry Pi 1/2/3, Odroid C1/2, Esp8266, Arduino, AVR, Android}
\cvcomputer{Syst\`emes}{Automatique, communication, localisation, intelligence embarqu\'ee, interface Homme-Machine (IHM)}
\cvcomputer{Plateformes}{UGV (Unmanned Ground Vehicle), UAV (Unmanned Aerial Vehicle), IOT}
\cvcomputer{Simulateurs}{Ardupilot SITL\developer, Gazebo\contributor}

\subsection{D\'eveloppement}
\cvcomputer{Langages}{C, C++, Python, Shell/Bash, GNU Make, JavaScript}
\cvcomputer{Frameworks}{ROS\contributor, OpenCV, Ardupilot\developer, HTML, Bootstrap, Websocket, Matlab}
\cvcomputer{Outils}{Git, Mercurial, Cmake, Waf, GitHub, Gitlab, Clion}
\cvcomputer{M\'ethodes}{TDD, Kanban, CI(Continuous Integration)}
\cvcomputer{Syst\`emes d'exploitation}{GNU/Linux (Debian, Ubuntu), Windows, Android}

\devnotes{D\'eveloppeur}{Contributeur}

\pagebreak

\section{Formations}

\tllabelcventry{2013}{0}{2013 -- }{Th\`ese CIFRE-D\'efense : M\'ecanismes de partage optimis\'e d'informations et de synchronisation entre syst\`emes embarqu\'es au sein d'un essaim de drones}{\href{http://www.lab-sticc.fr/en/index/}{Labsticc/Retdtech france}}{Rennes}{}{}

\tllabelcventry{2010}{2013}{2010 -- 2013}{Formation ing\'enieur}{\href{https://www.imt-atlantique.fr/}{\'Ecole des Mines de Nantes/ IMT Atlantique}}{Nantes}{Option: Automatique et Informatique Industrielle}{}

\tllabelcventry{2007}{2010}{2007 -- 2010}{Classe pr\'eparatoire scientifique aux grandes \'ecoles en fili\`ere Physique et Technologie}{\href{http://www.ldmraspail.fr/cpge-ptsi-pt.php}{Lyc\'ee Raspail}}{Paris}{Classe Etoile}{}

\tldatecventry{2007}{Baccalaur\'eat S}{\href{http://www.lm-st-cyr.fr/}{Lyc\'ee Militaire Saint Cyr l'\'Ecole}}{Saint Cyr l'\'Ecole}{Mention Assez Bien}{}

\section{Langues}
\cvlanguage{Anglais}{Comp\`etence professionnelle, IELTS: 7.0/9.0}{Pratique r\'eguli\`ere}
\cvlanguage{Portugais}{D\'ebutant}{Pratique occasionnelle avec ma femme Br\'esillienne}
%\cvlanguage{Russe}{D\'ebutant}{\'Etudi\'e \`a l'\'ecole et nombreux voyages dans les pays russophones}

\section{Centres d'int\'er\^et}
\cvhobby{Cuisine}{Amateur de bonne cuisine \`a la fois \`a la maison et \`a l'ext\'erieur}
\cvhobby{Bricolage}{Home staging et diverse am\'eliorations \`a la maison}
\cvhobby{Robotique}{Contribution \`a ROS et Ardupilot}
\cvhobby{Autres}{Voyages, lecture}

%\renewcommand{\listitemsymbol}{-} % change the symbol for lists

\end{document}

