\documentclass[11pt,a4paper]{moderncv}

\usepackage{PierreKancir}
\moderncvtheme[blue]{classic}
\usepackage[utf8]{inputenc}       % encodage à privilégier pour la portabilité et +
\usepackage[T1]{fontenc}      % encodage à privilégier pour la portabilité et +
\usepackage{eurosym}
\usepackage[french]{babel}
\usepackage{lmodern} 
\usepackage[scale=0.8,margin=0.9in]{geometry}
%\frenchbsetup{StandardLists=true} % regle les problemes de lists
% \linespread{0.9}

\usepackage[firstyear=2007,lastyear=2018]{moderntimeline}
%\tlmaxdates{1999}{2012}
% \tlwidth{0.8ex}

\title{Ingénieur Robotique}
\extrainfo{%
\linkedinsocialsymbol~\httplink{www.linkedin.com/in/pierre-kancir/fr}\\%
%\githubsocialsymbol~\httplink{www.github.com/khancyr}\\%
%Permis B%
}

% personal data
\firstname{Pierre}
\familyname{Kancir}
\address{130 Rue de Chatillon}{35200 Rennes}
\mobile{+33 (0) 6 67 64 64 84}
%\phone{}
%\fax{fax (optional)}                          % optional, remove the line if not wanted
\email{pierre.kancir@gmail.com}
%\homepage{}
%\photo[92pt]{identity}


%\nopagenumbers{}                             % uncomment to suppress automatic page numbering for CVs longer than one page
%----------------------------------------------------------------------------------
%            content
%----------------------------------------------------------------------------------
\begin{document}

%\setmainfont{Minion Pro}
%\setsansfont{Myriad Pro}

\hyphenpenalty=10000

\maketitle

\renewcommand{\labelitemi}{-- }

\section{Expérience professionnelle}

\cventry{2017 -- Aujourd'hui}{Ingénieur R\&D}{\href{http:///www.azurdrones.com/}{AzurDrones / Skeyetech}}{Mérignac}{}{%
PME spécialiste des services de captation et de traitement de données, conseil, formation et surveillance par drones.
\begin{itemize}
 \item Création de la seconde version de la solution de drone autonome avec station d'accueil ;
 \item Mise en place du service système embarqué : gestion des projets, installation d'un outil de versionning, d'intégration continue et de test ;
 \item Prototypage d'un drone en bus CAN avec UAVCAN et localisation par balise Ultra Wideband ;
 \item Renforcement par utilisation de simulations de la gestion et détection des pannes et des méthode de navigation du drone volant ;
 \item Encadrement d'une équipe de 4 personnes ;
\end{itemize}}

\cventry{2013 -- 2017}{Doctorant/ Ingénieur d'étude}{\href{http://www.retdtechfrance.fr/}{Retdtechfrance}}{Rennes}{}{%
TPE spécialisée en recherche et développement en mécatronique et constructeur de drones et de systèmes robotiques.
\begin{itemize}
 \item Création de systèmes robotiques et robotisés, systèmes multi robots multi-plateformes (air, terre, mer) ;
 \item Création d'IHM avec des technologie Web pour la commande des robots et l'affichage de vidéo en temps réel ;
 \item Gestion de projet, réponses à consultations, à appels d'offres publics, rédaction de documentation technique ;
 \item Participation à des projets de recherches: Essaim de drones {\href{https://recherche.telecom-bretagne.eu/daisie}{DAISIE}}, Bras robot gonflable {\href{http://www.warein-sas.com/c/innovation}{Bulle}.}
\end{itemize}}

\section{Compétences}

\subsection{Robotique}
\cvcomputer{Systèmes embarqués}{Type Rasberry Pi : Rasberry Pi 1/2/3, Odroid C1/2, Esp8266, Arduino, AVR, STM32, Android}
\cvcomputer{Systèmes}{Automatique, communication, localisation, navigation, intelligence embarquée, interface Homme-Machine (IHM)}
\cvcomputer{Plateformes}{UGV (Unmanned Ground Vehicle), UAV (Unmanned Aerial Vehicle), IOT}
\cvcomputer{Simulateurs}{Ardupilot SITL\developer, Gazebo\contributor}

\subsection{Développement}
\cvcomputer{Langages}{C, C++, Python, Shell/Bash, GNU Make, JavaScript}
\cvcomputer{Frameworks}{ROS\contributor, OpenCV, Ardupilot\developer, HTML, Bootstrap, Websocket, Matlab}
\cvcomputer{Outils}{Git, Mercurial, Cmake, Waf, GitHub, Gitlab, Clion, docker}
\cvcomputer{Méthodes}{TDD, Kanban, CI(Continuous Integration)}
\cvcomputer{Systèmes d'exploitation}{GNU/Linux (Debian, Ubuntu), Chibios, Windows, Android}

\devnotes{Développeur}{Contributeur}

\pagebreak

\section{Formations}

\tlcventry{2013}{2018}{Thèse CIFRE-Défense : Mécanismes de partage optimisé d'informations et de synchronisation entre systèmes embarqués au sein d'un essaim de drones}{\href{http://www.lab-sticc.fr/en/index/}{Labsticc/Retdtech france}}{Rennes}{}{}

\tllabelcventry{2010}{2013}{2010 -- 2013}{Formation ingénieur}{\href{https://www.imt-atlantique.fr/}{école des Mines de Nantes/ IMT Atlantique}}{Nantes}{Option: Automatique et Informatique Industrielle}{}

\tllabelcventry{2007}{2010}{2007 -- 2010}{Classe préparatoire scientifique aux grandes écoles en filière Physique et Technologie}{\href{http://www.ldmraspail.fr/cpge-ptsi-pt.php}{Lycée Raspail}}{Paris}{Classe Etoile}{}

\tldatecventry{2007}{Baccalauréat S}{\href{http://www.lm-st-cyr.fr/}{Lycée Militaire Saint Cyr l'école}}{Saint Cyr l'école}{Mention Assez Bien}{}


\subsection{Autres Expériences}
\cventry{2018}{Google Summer Of Code}{\href{https://summerofcode.withgoogle.com/archive/2018/projects/5702907169079296/}{Google}}{Encadrant}{}{%
Encadrement d'un étudiant au GSOC 2018 avec ArduPilot sur un projet de suivis de chemin par apprentissage avec un réseau de neurone pour un drone volant.}

\cventry{2017}{Google Summer Of Code}{\href{https://summerofcode.withgoogle.com/projects/\#6723821496172544}{Google}}{Étudiant}{}{%
Participation au GSOC 2017 avec ArduPilot.
\begin{itemize}
	\item Réécriture du code du Rover pour accepter différentes configurations et types de moteurs ;
	\item Implémentation de l'évitement d'obstacles ;
	\item Amélioration de la compatibilité avec ROS et Gazebo et création de tutoriels ;
	\item Réécriture de l'intégration continue.
\end{itemize}}

\cventry{2015 -- Aujourd'hui}{Développeur ArduPilot}{\href{ardupilot.org}{Ardupilot}}{}{}{%
ArduPilot est un projet open-source d'autopilote pour micro véhicules multi-plateformes.
\begin{itemize}
	\item Amélioration du code pour UGV et UAV;
 	\item Réécriture d'une partie du code en C++11 et Python3 et bugs fixes ;
 	\item Amélioration du simulateur SITL (Software In The Loop) et autotest;
 	\item Amélioration de la compatibilité avec ROS et Gazebo ;
 	\item Aide à la communauté et participation aux directions prises par le projet.
\end{itemize}}

\section{Langues}
\cvlanguage{Anglais}{Compétence professionnelle, IELTS: 7.0/9.0}{Pratique régulière}
\cvlanguage{Portugais}{Débutant}{Pratique occasionnelle avec ma femme Brésilienne}
%\cvlanguage{Russe}{Débutant}{étudié à l'école et nombreux voyages dans les pays russophones}

\section{Centres d'intérêt}
\cvhobby{Cuisine}{Amateur de bonne cuisine à la fois à la maison et à l'extérieur}
\cvhobby{Bricolage}{Home staging et diverse améliorations à la maison}
\cvhobby{Robotique}{Contribution à ROS et ArduPilot}
\cvhobby{Autres}{Voyages, lecture}

%\renewcommand{\listitemsymbol}{-} % change the symbol for lists

\end{document}

