\documentclass[11pt,a4paper]{moderncv}

\usepackage{PierreKancir}
\moderncvtheme[blue]{classic}
\usepackage[utf8]{inputenc}       % encodage à privilégier pour la portabilité et +
\usepackage[T1]{fontenc}      % encodage à privilégier pour la portabilité et +
\usepackage{eurosym}
\usepackage[french]{babel}
\usepackage{lmodern} 
\usepackage[margin=2.5cm]{geometry}
%\frenchbsetup{StandardLists=true} % regle les problemes de lists
% \linespread{0.9}
\AfterPreamble{\hypersetup{
  pdfauthor={Pierre Kancir},
  pdftitle={Pierre Kancir CV},
  pdfsubject={Pierre Kancir CV},
  colorlinks=true,
  urlcolor=blue,
  linkcolor=blue,
}}

\usepackage[firstyear=2007,lastyear=2021]{moderntimeline}
%\tlmaxdates{1999}{2012}
% \tlwidth{0.8ex}

%\title{Ingénieur Robotique}
\title{Ingénieur Etudes et Développements}
\extrainfo{%
\linkedinsocialsymbol~\httplink{www.linkedin.com/in/pierre-kancir/fr}\\%
\githubsocialsymbol~\httplink{www.github.com/khancyr}\\%
%Permis B%
}

% personal data
\firstname{Pierre}
\familyname{Kancir}
\address{5 rue Robelin}{35000 Rennes}
\mobile{+33 (0) 6 67 64 64 84}
%\phone{}
%\fax{fax (optional)}                          % optional, remove the line if not wanted
\email{pierre.kancir@gmail.com}
%\homepage{}
%\photo[92pt]{identity}


%\nopagenumbers{}                             % uncomment to suppress automatic page numbering for CVs longer than one page
%----------------------------------------------------------------------------------
%            content
%----------------------------------------------------------------------------------
\begin{document}

%\setmainfont{Minion Pro}
%\setsansfont{Myriad Pro}

\hyphenpenalty=10000

\maketitle

\renewcommand{\labelitemi}{-- }

\section{Expériences professionnelles}

\cventry{2020 -- Aujourd'hui}{Freelance en Robotique}{\href{https://www.hivebotics.fr}{Hivebotics}}{Rennes}{}{%
Bureau d'études en robotique mobile.
\begin{itemize}
    \item Développement de code.
    \item Formation.
    \item Services en robotique : tests, analyse de code et solution, veille technologique.
    \item Démonstrations d'usage de solutions Open Source : ROS, ArduPilot, BalenaOS.
\end{itemize}}

\cventry{2019 -- 2019}{Ingénieur R\&D}{\href{https://www.esoftthings.com/en/}{esoftthings}}{Cesson-Sévigné}{}{%
PME spécialisée le design de système Advanced Driver-Assistance Systems (ADAS) basé sur la vision et l'intelligence artificielle embarquée.
\begin{itemize}
    \item Création et évaluation d'un algorithme d'estimation de distance des détections effectuées sur un rétroviseur électronique (mono camera). Suite de tests automatisés sous ROS et Gazebo.
    \item Rédaction d'un manuel d'utilisation de la carte CV22 d'Ambarella.
    \item Implémentation d'un algorithme Driver Monitoring Système sur la carte CV22 pour une démonstration.
\end{itemize}}

\cventry{2017 -- 2019}{Ingénieur R\&D}{\href{http:///www.azurdrones.com/}{AzurDrones / Skeyetech}}{Mérignac}{}{%
PME spécialiste des services de captation et de traitement de données, conseil, formation et surveillance par drones.
\begin{itemize}
 \item Création de la première version de \href{https://www.youtube.com/watch?v=TsEw0Ob9nrs}{la solution de drone autonome avec station d'accueil} et mise en production ; 
 \item Mise en place du service système embarqué : gestion des projets, installation d'un outil de versionning, d'intégration continue et de test; %Encadrement du passage de 4 à 10 personnes ;
 \item \href{https://www.youtube.com/watch?v=PKJwuSgr23o&list=PL6sCNLbHuYxaLtItzADD5zHxixUtQPSvk&index=7&t=0s}{Renforcement par utilisation de simulations} de la gestion et détection des pannes et des méthodes de navigation du drone volant ;
 \item Définition, implémentation et tests des redondances du système ;
 \item Gestion des choix technologiques du projet drone autonome pour l'obtention de la première autorisation de vol autonome sans pilote en France, mise en place de projets de recherche pour la deuxième version.
\end{itemize}}

\cventry{2013 -- 2017}{Doctorant/ Ingénieur d'étude}{\href{http://www.retdtechfrance.fr/}{Retdtechfrance}}{Rennes}{}{%
TPE spécialisée en recherche et développement en mécatronique et constructeur de drones et de systèmes robotiques.
\begin{itemize}
 \item Création de systèmes robotiques et robotisés, systèmes multi-robots multi-plateformes (air, terre, mer) ;
 \item Création d'IHM avec des technologies Web pour la commande des robots et l'affichage de vidéo en temps réel ;
 %\item Gestion de projet, réponses à consultations, à appels d'offres publics, rédaction de documentation technique ;
 \item Participation à des projets de recherches: Essaim de drones {\href{https://recherche.telecom-bretagne.eu/daisie}{DAISIE}}, Bras robot gonflable {\href{http://www.warein-sas.com/c/innovation}{Bulle}}.
\end{itemize}}

\pagebreak

\section{Compétences}

%\subsection{Robotique}
\cvcomputer{Systèmes embarqués}{Linux SBC : iMx6, Beaglebone, Rasberry Pi, Odroid; Microcontrôleurs : Esp8266/32, Arduino, AVR, STM32; Android}
\cvcomputer{Systèmes}{Automatique, communication, localisation, navigation, intelligence embarquée, interface Homme-Machine (IHM)}
\cvcomputer{Plateformes}{UGV (Unmanned Ground Vehicle), UAV (Unmanned Aerial Vehicle), IOT}
\cvcomputer{Simulateurs}{\href{http://ardupilot.org/dev/docs/sitl-simulator-software-in-the-loop.html}{ArduPilot SITL}\developer, \href{http://gazebosim.org/}{Gazebo}\contributor}

%\subsection{Développement}
\cvcomputer{Langages}{C, C++, Python, Shell/Bash/Zsh, JavaScript}
\cvcomputer{Frameworks}{\href{https://www.ros.org/}{ROS}\contributor, \href{http://ardupilot.org}{ArduPilot}\developer, OPENCV, GStreamer, HTML, Bootstrap, Websocket, Matlab}
\cvcomputer{Outils}{Git, Valgrind, Make, Cmake, Waf, Docker, Travis, Github Actions, LibIIO}
\cvcomputer{Méthodes}{TDD, Kanban, CI(Continuous Integration)}
\cvcomputer{Systèmes d'exploitation}{GNU/Linux (Debian, Ubuntu), FreeRTOS, Chibios, Windows}

\devnotes{Développeur}{Contributeur}


\section{Formations}

\tlcventry{2013}{2018}{Thèse CIFRE-Défense : Méthodologie de conception de système multi-robots : de la simulation à la démonstration}{\href{http://www.lab-sticc.fr/en/index/}{Labsticc/Retdtech France}}{Rennes}{\href{https://drive.google.com/open?id=1kunx6TPXdK-vvV7pQeexgF_9M8Roi05N}{Lien du manuscrit}}{}

\tllabelcventry{2010}{2013}{2010 -- 2013}{Formation ingénieur}{\href{https://www.imt-atlantique.fr/}{école des Mines de Nantes/ IMT Atlantique}}{Nantes}{Option: Automatique et Informatique Industrielle}{}

\tllabelcventry{2007}{2010}{2007 -- 2010}{Classe préparatoire scientifique aux grandes écoles en filière Physique et Technologie}{\href{http://www.ldmraspail.fr/cpge-ptsi-pt.php}{Lycée Raspail}}{Paris}{Classe Etoile}{}

\tldatecventry{2007}{Baccalauréat S}{\href{http://www.lm-st-cyr.fr/}{Lycée Militaire Saint Cyr l'école}}{Saint Cyr l'école}{
%Mention Assez Bien
}{}


\subsection{Autres Expériences}
\cventry{2018}{Google Summer Of Code}{\href{https://summerofcode.withgoogle.com/archive/2018/projects/5702907169079296/}{Google}}{Encadrant}{}{%
%Encadrement d'un étudiant au GSOC 2018 avec ArduPilot sur un projet de suivi de chemin par apprentissage avec un réseau de neurone pour un drone volant.
}

\cventry{2017}{Google Summer Of Code}{\href{https://summerofcode.withgoogle.com/projects/\#6723821496172544}{Google}}{Étudiant}{}{%
Participation au GSOC 2017, programme de développement des logiciels open source mis en place par Google avec selection compétitive, avec ArduPilot.
%\begin{itemize}
%    \item Soumission d'un projet et acceptation d'un projet d'amélioration du support des drones terrestre (UGV).
%	\item Réécriture du code du Rover pour accepter différentes configurations et types de moteurs ;
%	\item Implémentation de l'évitement d'obstacles ;
%	\item Réécriture de l'intégration continue.
%\end{itemize}
}

\cventry{2015 -- Aujourd'hui}{Développeur ArduPilot}{\href{http://ardupilot.org}{ArduPilot}}{}{}{%
ArduPilot est un projet open-source d'autopilote pour micro véhicules multi-plateformes.
\begin{itemize}
	\item Amélioration du code pour UGV et UAV;
 	%\item Réécriture d'une partie du code en C++11 et Python3 et corrections de bugs ;
 	\item Amélioration du simulateur SITL (Software In The Loop) et des tests automatisés pour l'intégration continue;
 	\item Amélioration de la compatibilité avec ROS et Gazebo ;
 	\item Aide à la communauté et participation à la gouvernance du projet.
\end{itemize}}

\section{Langues}
\cvlanguage{Anglais}{Compétence professionnelle, IELTS: 7.0/9.0}{Pratique régulière}
\cvlanguage{Portugais}{Débutant}{Pratique occasionnelle avec ma femme Brésilienne}
%\cvlanguage{Russe}{Débutant}{étudié à l'école et nombreux voyages dans les pays russophones}

\section{Centres d'intérêt}
\cvhobby{Cuisine}{Amateur de bonne cuisine à la fois à la maison et à l'extérieur}
\cvhobby{Bricolage}{Home staging, homelab et diverses améliorations à la maison}

\end{document}

