\documentclass[11pt,a4paper]{moderncv}

\usepackage{PierreKancir}
%\usepackage[french]{babel}
%\frenchbsetup{StandardLists=true} % regle les problemes de lists
\linespread{0.9}

\title{Ing\'enieur Syst\`eme en Robotique}
\extrainfo{%
\linkedinsocialsymbol~\httplink{www.linkedin.com/in/pierre-kancir/fr}\\%
\githubsocialsymbol~\httplink{www.github.com/khancyr}\\%
%Permis B%
}


%\nopagenumbers{}                             % uncomment to suppress automatic page numbering for CVs longer than one page
%----------------------------------------------------------------------------------
%            content
%----------------------------------------------------------------------------------
\begin{document}

%\setmainfont{Minion Pro}
%\setsansfont{Myriad Pro}

\hyphenpenalty=10000
\maketitle

\renewcommand{\labelitemi}{-- }


\section{Comp\'etences}

\subsection{D\'eveloppement}
\cvcomputer{Langages}{C, C++, Python, Shell/Bash, GNU Make}{Web}{HTML, CSS, JavaScript, Bootstrap, WebSocket}
\cvcomputer{Frameworks}{ROS\contributor, OpenCV, Ardupilot\developer, Matlab}{Simulateurs}{Ardupilot SITL\developer, Gazebo\contributor}
\cvcomputer{Formats}{XML, YAML/JSON, Protobuf, Mavlink}{M\'ethodes}{D\'eveloppement orient\'e objet, TDD, Tests unitaires}
\cvcomputer{Gestion de sources}{Git, Mercurial}{Outils}{Cmake, Waf, GitHub, Gitlab, Clion}

\subsection{Administration syst\`eme et r\'eseau}
\cvcomputer{Syst\`emes d'exploitation}{GNU/Linux (Debian, Ubuntu), Windows, Android}{R\'eseaux et Bus}{Wifi, I2C, UART, USB, Ethernet, CAN(base), syst\`emes RF}

\subsection{Bureautique et outils}
\cvcomputer{Bureautique}{OpenOffice/LibreOffice, Microsoft Office, Gimp(base), Inkscape(base)}{Gestion de projet}{GanttProject, Kanboard}

\subsection{Robotique}
\cvcomputer{Syst\`emes embarqu\'es}{Type Rasberry Pi : Rasberry Pi 1/2/3, Odroid C1/2, Esp8266, Arduino, AVR, Android}{\'Electronique}{Pont en H, IMU, \'electronique de base , moteur CC, Capteurs de distance (lidar, sonar, IR), Syst\`emes de vision, Syst\`emes de positionnement (GPS, UWB)}
\cvcomputer{Syst\`emes}{\'Electronique de puissance, Asservissement, C\^ablage, Filtrage, Gestion de l'\'energie, Communication inter composant ou v\'ehicule}{Robot Types}{Rover, UGV, multirotors, IOT}

\devnotes{D\'eveloppeur}{Contributeur}

\pagebreak

\section{Exp\'erience professionnelle}

\tlcventry{2013}{0}{Doctorant/ Ing\'enieur d'\'etude}{\href{http://www.retdtechfrance.fr/}{Retdtechfrance}}{Rennes}{}{%
PME de 5 personnes sp\'ecialis\'ee en recherche et d\'eveloppement en m\'ecatronique et constructeur de drones et syst\`eme robotique.
\begin{itemize}
 \item Cr\'eation d'un simulateur multi robots \`a plusieurs niveaux bas\'es sur des composants r\'eels et d'un algorithme de classification des performances des syst\`emes multi robots ;
 \item Cr\'eation d'IHM avec des technologie Web pour la commande des robots et l'affichage de vid\'eo en temps r\'eel ;
 \item Cr\'eation de syst\`emes robotiques : drone volant, bou\'ee de p\^eche autonome, robot roulant, bras gonflable robotique, essaim de robots terrestres et volants, capteurs divers ;
 \item Gestion de projet ;
 \item R\'edaction de documentation technique ;
 \item Projets principaux :
 \begin{itemize}
  	\item Rapid Bulle :
 	\begin{itemize}
 		\item Cr\'eation d'une interface de commande (IHM et \'electronique) pour un bras robotique gonflable avec un retour vid\'eo sur tablette.
 	\end{itemize}
 	\item \OE{}il D\'eport\'e :
 	\begin{itemize}
 		\item Cr\'eation d'un drone volant d'inspection visuelle \`a commandes simplifi\'ees et s\'ecuris\'e avec un treuil automatique.
 	\end{itemize}
 	\item DAISIE \url{https://recherche.telecom-bretagne.eu/daisie} :
 	\begin{itemize}
 		\item R\'ealisation d'un essaim de robots h\'et\'erog\`enes (roulant et volant) bas co\^uts pour l'\'evaluation d'un algorithme de ph\'eromones ;
 		\item Objectif r\'ealis\'e :
 		\begin{itemize}
 			\item D\'emonstration d'un algorithme de d\'eplacement en essaim par carte de ph\'eromones ;
			\item Patrouille automatique ;
			\item D\'etection de nain de jardin ;
			\item Partage d'informations ;
			\item Reconfiguration de l'essaim ;
			\item Retransmission de flux vid\'eo.
		\end{itemize}
 	\end{itemize}
 	\item Bou\'ee pour la p\^eche :
 	\begin{itemize}
 		\item R\'ealisation d'une bou\'ee de p\^eche autonome, munis d'un sonar, devant s\'ejourner un temps important en mer et capable de communiquer \`a l'aide d'une liaison satellite sur la base d'un planning.
 	\end{itemize}
 	\item POSIVIZ :
 	\begin{itemize}
 		\item R\'ealisation d'un d\'emonstrateur pour le brevet de calcul de vitesse par vision mono camera pour les trains POSIVIZ.
 	\end{itemize} 
 \end{itemize} 	
\end{itemize}}

\subsection{Autres Exp\'eriences}
\tlcventry{2015}{0}{Developpeur ArduPilot}{\href{ardupilot.org}{Ardupilot}}{}{}{%
Ardupilot est un projet open-source d'autopilote pour micro v\'ehicules. Il pr\'esente l'avantage d'\^etre bas co\^ut, open source avec une forte communaut\'e et des industriels, et utilisable sur plusieurs types de plateformes. En effet, cet autopilote est utilisable sur des UGV (Unmanned Ground Vehicle), UAV (Unmanned Aerial Vehicle), aile volante, bateau, ROV. Cette plateforme est largement utilis\'ee dans le monde.
\begin{itemize}
	\item Am\'elioration du code pour UGV ;
 	\item R\'eecriture d'une partie du code en C++11 et Python3 et bugs fixes ;
 	\item Am\'elioration du simulateur SITL (Software In The Loop) ;
 	\item Am\'elioration de la compatibilit\'e avec ROS et Gazebo ;
 	\item Aide \`a la communaut\'e et participation aux directions prises par le projet.
\end{itemize}}




\pagebreak


\section{\'Education}

\tllabelcventry{2013}{2017}{2013 -- }{Th\`ese CIFRE-D\'efense : M\'ecanismes de partage optimis\'e d'informations et de synchronisation entre syst\`emes embarqu\'es au sein d'un essaim de drones}{\href{http://www.lab-sticc.fr/en/index/}{Labsticc/Retdtech france}}{Rennes}{}{Syst\`eme Multi Robots, Architecture distribu\'ee, Syst\`eme robotique, Publication scientifique, Simulation, Drone}

\tllabelcventry{2010}{2013}{2010 -- 2013}{Dipl\^ome ing\'enieur}{\href{https://www.imt-atlantique.fr/}{\'Ecole des Mines de Nantes/ IMT Atlantique}}{Nantes}{}{Contr\^ole-commande, Commande Optimale, Syst\`eme d'Etat, Traitement du signal, Microcontrôleurs, Robotique, Mod\'elisation des syst\`emes, Informatique industrielle}

\tllabelcventry{2007}{2010}{2007 -- 2010}{Classe pr\'eparatoire scientifique aux grandes \'ecoles en fili\`ere Physique et Technologie}{\href{http://www.ldmraspail.fr/cpge-ptsi-pt.php}{Lyc\'ee Raspail}}{Paris}{Classe Etoile}{Math\'ematiques, Physique, Chimie, Sciences Industrielles, Informatique, Lettres Philosophie, Langue Vivante}

\tldatecventry{2007}{Baccalaur\'eat S}{\href{http://www.lm-st-cyr.fr/}{Lyc\'ee Militaire Saint Cyr l'\'Ecole}}{Saint Cyr l'\'Ecole}{Mention Assez Bien}{}




\section{Langues}
\cvlanguage{Anglais}{Courant}{Pratique r\'eguli\`ere}
\cvlanguage{Portugais}{D\'ebutant}{Pratique occasionnelle avec ma femme Br\'esillienne}
\cvlanguage{Russe}{D\'ebutant}{\'Etudi\'e \`a l'\'ecole et nombreux voyages dans les pays russophiles}


\section{Centres d'int\'er\^et}
\cvhobby{Cuisine}{Amateur de bonne cuisine \`a la fois \`a la maison et \`a l'ext\'erieur}
\cvhobby{Bricolage}{Home staging et diverse am\'eliorations \`a la maison}
\cvhobby{Autres}{Voyages, lecture}

%\renewcommand{\listitemsymbol}{-} % change the symbol for lists

\end{document}

